\section{Introduction}
\label{sec:intro}
Magnetic Resonance Imaging is commonly used by medical community to safely and
 non-invasively probe the structure and function of human bodies. It has a 
profound impact both in clinical and research fields. 
%
MRI has a scan phase (data acquisition) and an image reconstruction phase. Short
scan time can increase scanner throughput and reduce patient discomfort, which
tends to mitigate motion-related artifacts. At the same time, high resolution of
the image is preferable, but it requires a longer scan time.
%
Sampling across a uniform grid (Cartesian trajectory sampling) enables quick and
efficient image reconstruction, but it requires a rather long scan time, which
limits the image resolution. Non-uniform sampling (non-Cartesian trajectory
sampling), instead, is faster and less sensitive to imaging artifacts, because
it incorporates some prior anatomical knowledge to scan specific areas with
higher resolution. However, this approach requires significantly more
computation for image reconstruction~\cite{stone2008accelerating}.\\
\\
Our accelerator focuses on the hardware acceleration of one key component of the image
reconstruction for non-Cartesian trajectory sampling: the computation of the Q
matrix~\cite{stratton2012parboil}. We will use ESP, a platform for
agile system-on-chip (SoC) design~\cite{esp-website, esp-release}, to design a 
hardware accelerator for the MRI-Q algorithm, and we integrate the accelerator in a 
full system-on-chip(SoC). 



\subsection{Motivation}
Heterogeneous systems architecture is the future trend. Hardware specialization
 can bring order-of-magnitude more energy efficiency.Designing an accelerator 
for computing MRI-Q matrix through ESP is a good way to learn ESP design methodology.
