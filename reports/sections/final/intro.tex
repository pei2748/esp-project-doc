\section{Introduction}
\label{sec:intro}

Magnetic Resonance Imaging is commonly used by medical community to safely and
non-invasively probe the structure and function of human bodies. Images
generated using MRI have a profound impact both in clinical and research
fields. MRI has scan phase (data acquisition) and image reconstruction
phase. Short scan time can increase scanner throughput and reduce patient
discomfort, which tends to mitigate motion-related artifacts. High resolution of
the image is preferable. Short scan time and high resolution conflict with each
other if sampling with the Cartesian trajectory in k-space on a uniform grid,
which allows image reconstruction to be performed quickly and
efficiently. However, the reconstruction of non-Cartesian trajectory sampling
data is faster and less sensitive to imaging artifacts caused by non-Cartesian
trajectories, but it increases computation
significantly~\cite{stone2008accelerating}. The computation for MRI-Q matrix is
an important algorithm used for image reconstruction with non-Cartesian
trajectory sampling~\cite{stratton2012parboil}.

\subsection{Motivation}

Heterogeneous systems architecture is the future trend. Hardware specialization
can bring order-of-magnitude more energy efficiency.Designing an accelerator for
computing MRI-Q matrix through ESP is a good way to learn ESP design
methodology.
