\section{Introduction}
\label{sec:intro}

Magnetic Resonance Imaging is commonly used by the medical community for safe
and non-invasive probing of the structure and function of human bodies. Images
generated using MRI have a profound impact both in clinical and research
fields.\\
\\
MRI has a scan phase (data acquisition) and an image reconstruction phase. Short
scan time can increase scanner throughput and reduce patient discomfort, which
tends to mitigate motion-related artifacts. At the same time, high resolution of
the image is preferable, but it requires a longer scan time.
%
Sampling across a uniform grid (Cartesian trajectory sampling) enables quick and
efficient image reconstruction, but it requires a rather long scan time, which
limits the image resolution. Non-uniform sampling (non-Cartesian trajectory
sampling), instead, is faster and less sensitive to imaging artifacts, because
it incorporates some prior anatomical knowledge to scan specific areas with
higher resolution. However, this approach requires significantly more
computation for image reconstruction~\cite{stone2008accelerating}.\\
\\
This work focuses on the hardware acceleration of one key component of the image
reconstruction for non-Cartesian trajectory sampling: the computation of the Q
matrix~\cite{stratton2012parboil}. For this purpose we use ESP, a platform for
agile system-on-chip (SoC) design~\cite{esp-website, esp-release}. With ESP, we
design a hardware accelerator for the MRI-Q algorithm, we perform a design-space
exploration (DSE), and we integrate the accelerator in a full system-on-chip
(SoC). Finally, we evaluate the accelerator performance by prototyping the full
SoC on FPGA.
