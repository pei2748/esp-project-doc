\section{Conclusion}

In this project, I have designed a set of accelerators using ESP to compute the
Q matrix for MRI image reconstruction. Any arbitrary input data sizes can find
an accelerator to work with. The accelerators are tested on the FPGA through
both bare-metal application and Linux application. I have also done different
levels of design space exploration. Customizing PLM designs and reducing the
fixed point precision can reduce both area and latency. Loop-unrolling and
loop-pipelining in the main computation-intensive loop can reduce latency
proportionally when compute kernel is dominating the whole procedure.

%% Heterogeneous SoC architectures  . Hardware specialization
%% can bring order-of-magnitude energy efficiency benefits. Designing an accelerator for
%% computing the MRI-Q matrix  through ESP is a good way to learn ESP design
%% methodology.

\subsection{Critical Aspects}

\begin{enumerate}

\item Use fixed-point datatype and various data conversions

\item Implement sine and cosine functions in SystemC.

\item Try different methods of optimization to reduce latency or decrease area.

\end{enumerate}

\subsection{Limitations}

We don't compare our acceleration result with the state-of-the-art MRI image
recustruction performance since we only accelerate one part of the iamge
reconstruction. The whole image recustruction of non-Cartician scanning contains
three steps: computing Q-matrix, computing the vector $\mathbf{F}^{H}
\mathbf{d}$, and finding the image iteratively via a conjugate gradient linear
solver. The first two steps are very similiar.~\cite{stone2008accelerating}
Since our accelerator only accelerate the first step, we can't evaluate the
reconstructed image quality and the performance with other works. Also, we don't
know the error tolerance of the first step.

\subsection{Future Work}
\begin{itemize}

\item Design the same set of accelerators with Vivado HLS C/C++ through ESP for
  the MRI-Q matrix computation. Compare the latency and area performance between
  the generated RTLs through two different design flows.

\item Compare with state-of-the-art

\item Related work

\item Implement the rest of the MRI application

\item Dynamically select between A0 and A1, not at design time.
  
\end{itemize}
