\section{Future Work}

\subsection{Accelerator Integration}

Finish the accelerator integration and prototpye on FPGA.

\subsection{Design Space Exploration}

There are two directions for design space exploration. The first is to reduce
 latency. The optimization methods I plan to do is as follows:

\begin{enumerate}

\item Loop unroll and pipeline.

\item Constrain latency.

\item Reduce fixed-point precision.

\item Add more ports to PLM to increase parallelism.

\item Optimize compute function. Optimize the computation of sine and cosine functions.

\end{enumerate}

The second direction is to reduce area. The following are the methods I plan to try.

\begin{enumerate}

    \item Use ping pong buffers to load a portion of data to do computation. 
      Load all the frequency-related variables (kx, ky, kz, phiR, phiZ) into PLM,
      but only load a portion of pixels (x,y,z). 

    \item In place storage of output. 

    \item Use different types of PLM blocks. frequency-related variables are 
      stored in one type of PLM with size numK. The pixel variables (x,y,z) 
      could be stored in a much smaller PLM block if using ping-pong buffers. 

    \item Reduce sine look-up table.  

    \item Reduce fixed-point storage. For now, data size is 32 bits. Later, I 
      will try to reduce this data width while guaranteeing the computation 
      accuracy. A smaller number of bits requires smaller storage capacity.

\end{enumerate}

I will generate a Pareto curve of my designs. Then, I will compare the FPGA 
running time with the software running time on FPGA to evaluate the performance 
of the accelerator.
