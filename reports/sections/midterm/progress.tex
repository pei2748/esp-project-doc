\vspace{-4mm}
\section{Progress}

\subsection{Programmer View Implementation}
The Parboil benchmark provides datasets to run the programmer view implementation, 
shown in Table \ref{tab-1}. For input data with image size ``small'' and ``large'', 
the execution takes 4.7 s and 18.7 s running on the google server (socp03), respectively. 
But for the ``128x128x128'' dataset, the run time is 186685 s $ \sim$ 2.2 days. The higher 
precision of the reconstructed image, the bigger the input data size, and the more computation.

\begin{table}[h!]
    \centering
    \begin{tabular}{c|c|c|c}
    \hline
       name  & image size & \# of pixels & K-space dimension  \\
    \hline
        small  & 32*32*32 & 32768 & 3072 \\
        large & 64*64*64 & 262144 & 2048\\
        128x128x128 & 128*128*128 & 2097152 & 2097152\\
        \hline
    \end{tabular}
    \caption{Datasets of MRIQ from Parboil Benchmark}
    \label{tab-1}
\end{table}

\subsection{Behavioral Simulation}

I generated skeleton code from the ESP template with two configurable registers: 
numX and numK, which corresponds to the \# of pixels and K-space dimemsion in 
Table \ref{tab-1}. Then I edited the code in the following files:\\
\begin{itemize}
\item \textbf{mriq/hw/src/fpdata.hpp} \\
Add fpdata.hpp to src/ folder. Typedef FPDATA. Defined functions used to do datatype
 conversions, for example, int2fp(), fp2int(), fp2bv(), bv2fp(). These functions are
 used in both the testbench and different processes of the accelerator design. In the
 testbench, data in floating-point type read from files is converted to fixed-point 
representation, then converted to sc\_bv to be transported through the network-on-chip. 
In accelerator, data in sc\_bv  type is converted to sc\_int to be stored in PLM. In 
computation phase, sc\_int data is converted to fixed -point representation.. After 
computation finished, The fixed-point data needs to be converted back to sc\_int in 
order to be stored in PLM. Then, it is further converted to sc\_bv to be transported 
back to the testbench for validation. In testbench, sc\_bv is converted to fixed-point 
representation. and further converted to floating point type to be compared with the 
golden output data.

%%%%%%%%%%%%%%%%%%%%%
\item \textbf{mriq/hw/src/mriq.hpp} \\
I declared 10 PLMs, with name plm\_x, plm\_y, plm\_z, plm\_kx, plm\_ky, plm\_kz, 
plm\_phiR, plm\_phiI, plm\_Qr,  plm\_Qi, and declared functions \textit{mySinf}, 
\textit{computeQ}, \textit{load\_one\_data}, and \textit{store\_one\_data}, used
 in mriq.cpp.

%%%%%%%%%%%%%%%%%%%%%
\item \textbf{mriq/hw/src/mriq.cpp} \\
Rewrote the processes \textit{load\_input},\textit{compute\_kernel}, \\
and  \textit{store\_output}. 
%%%%%%%%%%%%%%%%%%%%%
\item \textbf{mriq/hw/src/mriq\_functions.hpp} \\
Wrote function \textit{ComputeQ} which is the key computation part of Q matrix, 
showed in Fig.\ref{fig-1}. Wrote function \textit{mySinf}, which is to compute 
sine. In sine function, first convert an input value to $0\sim \pi/2$ range, 
then find the closest data point in the sin\_table. For now, I use the interpolation 
method to get the sin(x). \textit{load\_one\_data} function is to load one variable
 into one PLM. Similarly, \textit{store\_one\_data} is to store one variable into 
one PLM.

%%%%%%%%%%%%%%%%%%%%%%%%%%%%%%
\item \textbf{mriq/hw/tb/system.hpp} \\
Declared the paths for input file and golden output file. Modified the 
SC\_HAS\_PROCESS constructor. Declared parameters and functions used in 
system.cpp file.

\item \textbf{mriq/hw/tb/system.cpp} \\
Rewrote functions \textbf{\textit{load\_memory}}, \textbf{\textit{dump\_memory}},
 and \textbf{\textit{validate}}. Added function \textbf{\textit{inputData}} to 
read input data from file, and \textbf{\textit{outputData}} to read golden 
output data from file.

\item \textbf{mriq/hw/tb/sc\_main.cpp} \\
Added the paths and names of the input file and the golden output file.
\end{itemize}

Behavioral simulation succeeded when running the whole dataset (32*32*32). 
It takes a few hours to run since the fixed-point datatype is not native to c/c++. 

\subsection{RTL generation}
I have generated RTL successfully by running ``make mriq\_stratus-hls'' in the working
 folder. In this process, all the print sentences in the files in the src/ 
folder should be deleted to avoid that the ESP\_REPORT causes errors in this 
process. In every for-loop, there should be a wait() sentence after the ``for
-statement'' to allow Stratus to perform loop unrolling.

\subsection{RTL Simulation}
By running an accelerated simulation method, I have tested the generated RTL
 without error. Instead of computing the whole dataset, which has 32K pixels, 
I edited the code to compute only the first 4 pixels in the input dataset and
 reduced the k-space size from 3K to 16. The simulation of RTL succeeded. 
My accelerating method of simulation is to edit the default value of (numX, numK) 
from (32768, 3072) to (4, 16) both in tb/system.hpp and src/mriq\_conf\_info.hpp. 
