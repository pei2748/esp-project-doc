\section{Specification}
The algorithm for computing the MRI Q matrix, shown in Fig.~\ref{fig-1}. The main 
accelerating direction is to unroll the for-loops. The programmer view algorithm
 C code and input data set come from the Parboil benchmark suite~\cite{stratton2012parboil}.


%% In case you need to insert one or multiple figures with more complex layout,
%% you can use minipage. Save it to PDF and use the following
%% code. Remember to push the figure to the repository!
%% You can refer to a figure in the text with Fig.~\ref{fig:my_lable}
%% The letter in square brackets determines the position with respect to the
%% page: t=top; b=bottom; h=here
%% Change the minipage or the box width to adjust the size and the margins.
%% \begin{figure}[t]
%%   \begin{minipage}{\linewidth}
%% \centering
%%     \resizebox{\textwidth}{!}{
%%       \includegraphics{<path to figure>}
%%     }
%%   \end{minipage}
%%   \caption{Add a caption}
%%   \label{fig:my_label}
%% \end{figure}

\subsection{Assessment}
The goal of the assessment is to return correct Q matrix implemented on FPGA 
and compare the executing time of CPU and the accelerator designed. Also, 
several different designs will be implemented on FPGA to explore the design space.

\subsection{Milestones}\label{sec:arch}
\label{sec:milestones}

\begin{enumerate}

\item Project proposal (by Feb. 5). -- Done!

\item Reference application (by Feb. 12). -- Done!

\item ESP tutorials: \emph{``How to design a single-core SoC''}, and \emph{``How
  to design an accelerator in SystemC (Cadence Stratus HLS)''} (by Feb. 19). -- Done!

\item Skeleton generation for accelerator and test apps (by Feb. 19) -- Done!

\item Accelerator and test apps customization (by Mar. 11) -- Done!
 
\item SoC evaluation with RTL simulation (by Mar. 18) -- Done!

\item SoC evaluation with FPGA prototyping (by Mar. 25) -- Done!

\item Midterm presentation and report (by Mar. 25)

\item DSE and optimization (by May 13)

\item Final presentation and report (by May 13)

\end{enumerate}

\subsection{Critical Aspects}
\begin{enumerate}
\setlength\itemsep{-0.15em}
\item Implement sine and cosine functions in the innermost for-loop.
\item Try different methods of optimization to reduce latency or decrease area.
\end{enumerate}

\begin{figure}[t]
\centering
\captionsetup{justification=centering, format=hang}
\includegraphics[width=0.85\columnwidth]{figures/algorithm}
\caption{Algorithm for computing MRI Q matrix~\cite{stone2008accelerating}}
\label{fig-1}
\end{figure}
