\section{Introduction}
\label{sec:intro}
Magnetic Resonance Imaging is commonly used by the medical community to safely 
and non-invasively probe the structure and function of human bodies. Images 
generated using MRI have a profound impact both in clinical and research fields.
 The reconstruction of non-Cartesian trajectory sampling data is faster and 
less sensitive to imaging artifacts caused by non-Cartesian trajectories than 
sampling in Cartesian space, but it increases computation significantly
~\cite{stone2008accelerating}. The computation for the MRI-Q matrix is an important 
algorithm used for image reconstruction with non-Cartesian trajectory 
sampling~\cite{stratton2012parboil}. So accelerating MRI-Q matrix 
computation helps MRI to benefit human beings.

\subsection{Motivation}
Heterogeneous systems architecture is the future trend. Hardware specialization 
can bring order-of-magnitude more energy efficiency. Designing an accelerator 
for computing the MRI-Q matrix through ESP is a good way to learn the ESP design
 methodology.

