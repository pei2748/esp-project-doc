\documentclass{sig-alternate}
%[preprint]
% The following \documentclass options may be useful:
%
% 10pt          To set in 10-point type instead of 9-point.
% 11pt          To set in 11-point type instead of 9-point.
% authoryear    To obtain author/year citation style instead of numeric.

\makeatletter
\def\@copyrightspace{\relax}
\makeatother

\usepackage[nynorsk,british,UKenglish,USenglish,english,american]{babel}

\usepackage{graphicx}
\usepackage{tikz}
%\usepackage{gnuplot-lua-tikz}
\usepackage{color}
\usepackage{tabularx}
\usepackage{fixltx2e}
%% \usepackage{dblfloatfix}
\usepackage{varwidth} % http://ctan.org/pkg/varwidth
\usepackage{listings}
\usepackage{url}
\usepackage{balance}
\usepackage{amsmath}
\usepackage{enumitem}
\usepackage{caption}

\lstset{
%  backgroundcolor=\color{yellow!20},%
  basicstyle=\small\ttfamily,%
  numbers=left, numberstyle=\tiny%
}

\newtheorem{thm}{Problem}
\newdef{definition}{Definition}
\DeclareMathAlphabet{\mathpzc}{OT1}{pzc}{m}{it}
\newtheorem{theorem}{Theorem}[section]
\newtheorem{lemma}[theorem]{Lemma}

\usepackage{xcolor}
\usepackage{framed}
\usepackage{amssymb,amsmath}
\usepackage{ifxetex,ifluatex}
\usepackage{fancyvrb}
\usepackage{comment}

\begin{document}

\title{\Large\bf Design an accelerator for Computing MRI-Q Matrix}
\subtitle{\normalsize COMS E6868 - Embedded Scalable Platforms - Spring 2020}

\numberofauthors{1}
\author{
\alignauthor
Pei Liu\\
\vspace{0.2cm}
       \email{pl2748@columbia.edu}
}

\vspace{-2cm}

\maketitle

\vspace{-2cm}

\begin{abstract}
{\small\em
  I plan to design an accelerator to calculate the Magnetic Resonance Imaging Q matrix through SystemC and HLS tool Stratus and implement this design on FPGA. Then I plan to explore the design space to get a thorough understanding of the methodology of designing accelerators though ESP. 
}
\end{abstract}

\section{Introduction}
\label{sec:intro}
Magnetic Resonance Imaging is commonly used by medical community to safely and non-invasively probe the structure and function of human bodies. Images generated using MRI have a profound impact both in clinical and research fields. MRI has scan phase (data acquisition) and image reconstruction phase. Short scan time can increase scanner throughput and reduce patient discomfort, which tends to mitigate motion-related artifacts. High resolution of the image is preferable. Short scan time and high resolution conflict with each other if sampling with the Cartesian trajectory in k-space on a uniform grid, which allows image reconstruction to be performed quickly and efficiently. However, the reconstruction of non-Cartesian trajectory sampling data is faster and less sensitive to imaging artifacts caused by non-Cartesian trajectories, but it increases computation significantly~\cite{stone2008accelerating}. The computation for MRI-Q matrix is a crucial part used for image reconstruction with non-Cartesian trajectory sampling~\cite{stratton2012parboil}.

\subsection{Motivation}
Heterogeneous systems architecture is the future trend. Hardware specialization can bring order-of-magnitude more energy efficiency.Designing an accelerator for computing MRI-Q matrix through ESP is a good way to learn ESP design methodology.

\section{Specification}
The algorithm for computing MRI Q matrix, shown in Fig.~\ref{fig:esp}. The main accelerating direction is to unroll the for-loops. The programmer view algorithm C code and input data set come from Parboil benchmark suit~\cite{Rub1}.


%% In case you need to insert one or multiple figures with more complex layout,
%% you can use minipage. Save it to PDF and use the following
%% code. Remember to push the figure to the repository!
%% You can refer to a figure in the text with Fig.~\ref{fig:my_lable}
%% The letter in square brackets determines the position with respect to the
%% page: t=top; b=bottom; h=here
%% Change the minipage or the box width to adjust the size and the margins.
%% \begin{figure}[t]
%%   \begin{minipage}{\linewidth}
%% \centering
%%     \resizebox{\textwidth}{!}{
%%       \includegraphics{<path to figure>}
%%     }
%%   \end{minipage}
%%   \caption{Add a caption}
%%   \label{fig:my_label}
%% \end{figure}

\subsection{Assessment}
Return correct Q matrix implemented on FPGA, and compare the executing time of CPU and the accelerator designed. Also, several different designs will be implemented on FPGA to explore the design space.

\subsection{Milestones}\label{sec:arch}
\label{sec:milestones}

\vspace{-0.1in}
\begin{enumerate}
\setlength\itemsep{-0.15em}
  \item Analysis of the algorithm and the programmer's implementation in C (by Feb. 19)
  \item Learning two tutorials: "How to design an accelerator in SystemC (Cadence Stratus HLS)" and "How to design a single-core SoC"\cite{esp1}. (by Feb. 28)
  \item High-Level-Synthesis implementation in SystemC (by Mar. 11)
  \vspace{-2mm}
       \begin{itemize}
            \item Transform programmer view algorithm to HLS-ready SystemC
            \item Behavioral simulation
       \end{itemize}

  \item Baremetal application and linux application design. Evaluation on an FPGA platform (by Mar. 25)
  \item Mid-term presentation and report (by Mar. 25)
  \item Initial Design Space Exploration (by Apr. 15)
  \item Enhanced Design Space Exploration (by Apr. 22)
  \item Final refinement and analysis (by May 5)
  \item Design the same accelerator with Vivado HLS flow.
  \item Final presentation ($\sim$ May 11) and report ($\sim$ May 15)
\end{enumerate}

\subsection{Critical Aspects}
\begin{enumerate}
\setlength\itemsep{-0.15em}
\item Implement sine and cosine functions in for-loop.
\item Try different methods of optimization to reduce latency or decrease area.
\end{enumerate}

\begin{figure}[t]
\centering
\captionsetup{justification=centering, format=hang}
\includegraphics[width=0.85\columnwidth]{figure/algorithm-proposal.png}
\caption{Algorithm for computing MRI Q matrix~\cite{stone2008accelerating}}
\label{fig:esp}
\end{figure}


%Bibliography
{\small
\balance
%\bibliographystyle{abbrv}
\bibliographystyle{unsrt}
\bibliography{ref}
}



% If you need to add an appendix with large figures or table use the following
% code:

%% \newpage
%% \onecolumn{
%% \centering
%% \section*{APPENDIX}
%% \vspace{0.5in}

%% % Add your Appendix text and figures here.

%% }

\end{document}
